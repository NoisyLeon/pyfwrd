%-=-=-=-=-=-=-=-=-=-=-=-=-=-=-=-=-=-=-=-=-=-=-=-=-=-=-=-=-=-=-=-=-=-=-=
\documentclass[11pt]{article}
\usepackage{times}
\usepackage[fleqn]{amsmath}
\usepackage{amsfonts}
\usepackage{pstricks,epsfig,psfrag,alltt,a4wide}
\usepackage{chicago}
\usepackage{symbol}
%-=-=-=-=-=-=-=-=-=-=-=-=-=-=-=-=-=-=-=-=-=-=-=-=-=-=-=-=-=-=-=-=-=-=-=
%----------------
\begin{document}
%----------------
Author: Yann Capdeville\\
email: capdevil@ipgp.jussieu.fr\\
Version 0.0, Paris, August 2005.
%%%%%%%%%%%%%%%%%%%%%%%%%%%%%%%%%%%%%
\section{Prelude and warning}
%%%%%%%%%%%%%%%%%%%%%%%%%%%%%%%%%%%%%
The is an attempt to provide a (poor) documentation for 
different programs to compute normal modes and synthetic
seismograms with normal mode summation in spherically symmetric Earth models.
The present version (0.0) of these programs may  not have been used
extensively. For example the normal mode summation code ({\tt nms})
is not the one I use usually to test and validate Spectral Element
codes in the Earth, but have been specially developed for the occasion of the
spice meeting from a time reversal code I wrote this year. This is to
say that a lot of errors and bugs may still be present in the codes. 
Be warned if you used these codes for your own work (but if you do, a
little thank you in the acknowledgment of your corresponding paper
would be appreciated) . 
%++++++++++++++++++++++++++++++++++++
\subsection{Documentations? About what?}
%++++++++++++++++++++++++++++++++++++
\begin{itemize}	
\item {\tt minosy} (alias {\tt yannos}): this program computes
normal modes (eigenfunctions and eigenfrequencies) for a given 
spherically symmetric Earth model in a given frequency band.
An eigenfunction for given eigenfrequency $\omega_{q,n,\ell}$ is written as
$$\bu(\br)=\left[{_n}U_\ell(r) \be_r+{_n}V_\ell(r)
\bnabla_1-{_n}W_\ell(r)(\be_r\times\bnabla_1)
\right]Y_\ell^m(\theta,\phi)$$ where $q$ stands for spheroidal or
toroidal, $n$ is the radial order, $\ell$
the angular order and $m$ the azimuthal order. Note that, because of
the spherical symmetry, eigenfunctions and eigenfrequencies do not
depend on $m$. $\bnabla_1 f(\br)
=\partial_\theta f(\br)\,
\be_\theta+(\sin\theta)^{-1}\partial_\phi\,f(\br)\,\be_\phi$.
$Y_\ell^m$   is the spherical harmonic. The mode normalization is such
that 
\begin{equation*}
\int _0^{\rt} \rho(r) \left( {_n}U_\ell^2(r) + \ell (\ell +1)
{_n}V_\ell^2(r)\right) r^2 \, 
\ud r =
\int _0^{\rt} \rho(r)\,\ell (\ell +1)  {_n}W_\ell^2(r) \, r^2 \ud r = 1\,,
\end{equation*}
where $\rt$ is the Earth radius. {\tt minosy} computes
$\omega_{q,n,\ell}$, ${_n}U_\ell(r)$, ${_n}V_\ell(r)$ and
${_n}W_\ell(r)$ for a given Earth model (note that we have
${_n}W_\ell(r)=0$ for spheroidal modes and
${_n}U_\ell(r)={_n}V_\ell(r)=0$ for toroidal modes).
\item {\tt stock\_boule}: Same as {\tt minosy}, but for homogeneous
sphere only. The eigenfunctions are computed analytically with Bessel
functions, which can be sometimes be interesting for accuracy reasons.  
\item {{\tt nms}}: this program computes synthetic seismograms for 
a given list of receivers and sources using the normal mode
catalog produced by {\tt minosy}.
\item {\tt get\_fctp}: little program to extract a given normal mode
from the output of {\tt minosy}. It can be useful to visualize 
the eigenfunctions.
\item {\tt generate\_prem}:  little program to generate different 
kind of Earth models to be used by {\tt minosy}.
\end{itemize}	
%++++++++++++++++++++++++++++++++++++
\subsection{References}
For normal mode computations, it can be useful to read 
\shortciteN{Takeuchi_Saito_1972}, \shortciteN{Saito_1988},
\shortciteN{Woodhouse_1988} and for normal mode summation,
e.g. \shortciteN{Woodhouse_Girnius_1982}. A good review of the 
subject can be found in \shortciteN{Dahlen_1998}.
\vspace{-1cm}
\bibliographystyle{chicago}
\def\refname{}
\bibliography{biblio}
\section{Compiling the package}
%++++++++++++++++++++++++++++++++++++
You will need a Fortran 90 compiler to compile the package.
Run first the {\tt configure} script which will generate the
{\tt flags.mk} to be used by the different makefile.
It will ask you  what comiler do you have:
\begin{itemize}	
\item {\tt intel} for intel compiler (ifort)
\item {\tt  g95}  for GNU fortran 95 compiler 
\item{\tt pgf} for portland compiler
\item{\tt xlf90} for IBM compiler (e.g. for mac?)
\item{\tt dec} for DEC ALPHA compiler
\item{\tt sun} for SUN
\end{itemize}	
 If you need
another compiler, you can easily add an new entry in the {\tt
configure} script. Warning: if you use {\tt g95}, the {\tt -O} level
is set to {\tt 0} otherwize {\tt minosy }
has some problems. Even with {\tt -O0}, results are not really
accurate (it may also mean that some parts of the code are still not
standard enough ... should debug that ... one day). I guess it will be
fixed with futur versions of g95. The version I m using now is: {\tt
gcc version 4.0.0 20050129 (experimental)   (g95!) Apr 13 2005}

Run {\tt make} or {\tt gmake} at the top
of the package tree and you should get all the executable files
in {\tt bin/} directory. If you don't ... well you will have to work 
a little bit more than expected or send me an email (try the first
option first!).

\section{Programs use}
An example of each input file (``{\tt .dat}'') is given in the {\tt
examples} directory.
\subsection{generate\_prem}
{\tt generate\_prem} produces some different Earth model ascii files
to be used as an input by {\tt minosy}. When using this program, 
an number of layers of 500 may be a good idea for period down to 50~s.
It may be interesting to increase this number of layers for higher
frequencies to get a good accuracy for attenuation, group velocity and
Rayleigh coefficients.
The output of format {\tt
generate\_prem} (and input of {\tt minosy}) is the following:
\begin{itemize}	
\item line 1: name of the model
\item line 2: {\tt ifani}, reference frequency for attenuation
dispersion, {\tt ifdeck}. It is a good idea to keep {\tt ifani} to
one. Only {\tt 
ifdeck=1} will be consider in the document.
\item line 3: total number of model lines, index of the ICB, index of
the OCB, number of line for the ocean (can be set to 0).
\item line 4 and more:
radius, density, $V_{pv}$, $V_{sv}$, $Q_{kappa}$, $Q_{shear}$, 
$V_{ph}$,  $V_{sh}$, $\eta$.
The format (in fortran) must be: {\tt f8.0, 3f9.2, 2f9.1, 2f9.2, f9.5}

\end{itemize}	

\subsection{MINOS}
\subsubsection{History}
{\tt minosy} (I used to call it {\tt yannos}; I know, this is a stupid name
that is why I changed that for the spice meeting) is my version of the
MINOS program. MINOS is a program with long history which I only know
partially. As I understand, at least 3 persons contribute to its
development more than twenty years ago: F. Gilbert,  
J. Woodhouse and G. Masters, but it is unclear to me who exactly did
what. It seems that the first version of the program was {\tt eos} and
then MINOS. Another normal modes  program with a lot of common featurs
with MINOS from J. Woodhouse is OBANI. In 1997 I started working on the coupling of spectral elements
with  modal  solution. To compute the modal solution, I started  from 
MINOS.  Because most of MINOS was written with an old
Fortran standard (a lot of ``goto'' and indexed ``if'') I had to
translate many parts of the program to more standard Fortran because
I wasn't able to work efficiently with the older standard. Finally,
after finishing working with the coupling of spectral elements
with  modal  solution, I also started to use my modified MINOS code
to compute classical normal mode solution and it appears that this
version is good enought. Compared to the original
version, I didn't introduce anything new, it is even less efficient
than the original version (not for the MPI case). The original version
had some little bugs that make it difficult to use to get the full wavefield 
at relatively high frequency. If you look at {\tt minosy},
you'll see that it is messy. That is true, but it is a mess that I'm
able to understand which wasn't the case for MINOS. Finally, you
should always be careful when using this program as I may have
introduced bugs that weren't in the older version (it has been the case
many times in the past). I'm sure that J. Woodhouse version would do a
better job that this one.
\subsubsection{Running the code}
There is only one input file, {\tt yannos.dat} that must be in the
running directory. Here is an example (where line number have been
added and should be removed for an actual run):
\begin{alltt}
\small
 1 #input earth model file (e.g. created with generate_prem) :
 2 prem
 3 #output information
 4 per_premR
 5 #prefix for output eigunfunction files
 6 fct_premR
 7 #type code (3=spheroidal, 2=toroidal, 0=both)
 8 3
 9 #precision (2) and switch off gravity perturbation frequency (eps1,eps2,wgrav)
10 1.E-10 1.E-10 10.
11 #lmin, lmax, fmin, fmax, nmax:
12 0 1500 0.1 15. 3000
13 ############if you need to output only some radius layers of eigenf
14 #number of layer for output:
15 1
16 #radius layers
17 3480000. 6371000.
18 #### computations flags. 
19 # force fmin (usually F):
20 F
21 # Cancel Gravity (usually F)
22 F
23 # never use start level (usually F):
24 F
25 #use T ref (usually T):
26 T
27 #Check modes (usually F):
28 F
29 # use_remedy  awkward modes (usually T):
30 T
31 # rescue  (try to do something if missing a mode ) (usually T):
32 T
33 #restart (to restart in case of crash during computation
34 F
35 #force_systemic_search
36 F
37 # keep_bad_modes
38 F
39 #  modout_format (ipg, ucb, olm) (ipg is the standard for me)
40 ipg
41 # seuil_ray
42 0.0001
43 # l_startlevel
44 0
\end{alltt}	
\begin{itemize}	
\item line 2: input model (e.g. output from {\tt generate\_prem}). 
\item line 4: normal mode catalog information output in ASCII.
It will give the eigenfrequency, group velocity, attenuation for each
mode. The last two column are the Rayleigh Quotient which give an
indication of the accuracy (the smallest is the best) and a logical
that is .true. if the mode is an inner core mode (not always
accurate). (Rayleigh Quotient: if $\bH$ is the elastic operator of the
wave equation and if $\bu_k$ the eigenmode associated with the
eigenfrequency $\omega_k$, therefore $(\bu_k,\bH\bu_k)=\omega_k^2$ (
$(.,.)$ is  the inner product). In
this file, the Rayleigh Quotient is $1-(\bu_k,\bH\bu_k)/\omega_k^2$ 
and should be equal to a small value if $\bu_k$ is indeed a mode 
associated with the frequency $\omega_k$). 
\item line 6: prefix of the eigenfunction output. If the ``ipg''
format is used, on binary file ({\tt .direct}) and one ascii file ({\tt
.info}) will be created from this prefix.
\item line 8: if 2 is entered, toroidal modes will be computed, if 3
spheroidal modes will be computed. If 0, both toroidal and spheroidal
modes will be computed. In the output files, a ``S'' will be hadded
for the spheroidal files and ``T'' for the toroidal ones.
\item line 10: the two first numbers are respectively integration
precision and precision on the eigenfrequency search. The last number
is the frequency (in mHz) after which the gravity potential
perturbation is not computed anymore. If you set this last value to 0,
you will be in the Cowling approximation.
\item line 12: the normal modes will be computed between  angular orders
{\tt lmin} to {\tt lmax} and the frequency {\tt fmin} to {\tt fmax} (in
mHz). {\tt nmax} in the maximum number of overtones that will be
computed for each angular order. If you need the whole wavefield
solution for a given frequency band, set {\tt lmin} to 0 and {\tt
lmax} and {\tt nmax} to large value (e.g. 5000). If you need only the
fundamental branch,  set {\tt nmax} to 1.
\item line 14 to 17: allow to choose one or more depth range where
eigenfunctions are stored. Usually one layer for 700km depth
up to surface is enough.
\item {\tt force\_fmin} (line 20): In normal use, the start frequency
search increases with the  angular degree l.  But for some situations, it can make sense
             to desactivate this feature by saying "true". Usual answer 
             is "false"
\item {\tt cancel\_gravity} (line 22): "true" means that ALL gravity terms will be set to 0.0
             Usual value is ".false."
       Warning: if that option is set, the starting frequency search
       shouldn't be to low because it makes failed the alternate
       Bessel subroutine  I'm using is that case. 
       The consequence of this, is that the 0S1 and 1S1 may not be 
       found. (for prem, fmin$<$0.05mHz is not good)
\item {\tt never\_use\_startlevel} (line 24): startlevel is subroutine that determine the integration 
             starting point in depth. 
             Usual value is ".false."
             If you have problems with some strange model (e.g. full homogeneous models)
             it may be a good idea to set this to ".true."
                
\item {\tt use\_tref} (line 26)     : Use the reference period of the model (for  example: prem
                1 s)
	    Usual value is ".true." 
\item {\tt check\_modes} (line 28)  : try to cancel some bad modes (some Stonley and inner core modes)
                Don't use it unless you know what you are doing.
              Usual answer is ".false."
\item {\tt use\_remedy} (line 30)  : remedy was build to recompute some modes like inner core modes.
              The original minos has it, but some problems come from this feature.
            Usual, I don't use it, but up to you.
            No Usual value.
\item {\tt rescue} (line 32) : Try to do something if the bracketing of modes fails. Us a systematic 
	    search for some frequency. It's not very efficient, and it can't heart
            anyway.
            Usual value is ".true." 
\item {\tt modout\_format} (line 34): 
\begin{itemize}	
\item ucb: UC Berkeley
\item                 olm: old minos
\item                 ipg: my format
\end{itemize}	
 Only "ipg" is implemented in the MPI version, but easy to do
yourself. The {\tt nms} program uses the ipg format.
\item {\tt l\_startlevel} (line 44): If you use "startlevel", specify after which "l" it should do so.
0 is the standard value.
\item {\tt seuil\_ray } (line 38) : cancels the mode if its Rayleigh
quotient is larger than this value. 
             $10^{-4}$ can be a good choice, but it really depends of what you are doing.
             It's really up to you. Information about the discarded
modes
will be found in the {\tt .lost} file.
\item {\tt restart}: allows to restart after a crash and avoids to
restart from start each time. For debug only.
\item {\tt force\_systemic\_search}: an alternate frequency search:
don't use it.
\end{itemize}	
\subsection{yannos\_MPI version}
For large frequency band, the computation can take time.
In that case and if you have access to a parallel computer, the MPI
version of {\tt minosy}, {\tt yannos\_MPI} can be useful.
To compile this version, you will need to set the MF90 variable in the
flag.mk to the proper mpi f90 script (e.g. mpif90 for mpich) and
execute {\tt make yannos\_MPI} in the  {\tt yannos/src} directory.
\subsection{stock\_boule}
{\tt stock\_boule.dat} does the same job as {\tt minosy} but for
homogeneous sphere only using Bessel functions to compute the 
solutions (see \shortciteNP{Takeuchi_Saito_1972}).
There is a single input file:
{\tt stock\_boule.dat}:
\begin{alltt}
 1                Lmax= 0500
 2 number of layers   = 0600
 3 maximum of overtone= 0401
 4      rho (kg m^-3) = 03000.0
 5           vp (m/s) = 08000.0
 6           vs (m/s) = 06000.0
 7  bowl radius(m)    = 6.371E+06
 8  frequency minimum = 1.000E-08
 9  frequency maximum = 0.300E-02
10 prefix file Rayleig= fctS
11 prefix file Love   = fctT
\end{alltt}	
where the format must be respected (the first 15 characters of each
lines won't be used as an input). 
\begin{itemize}	
\item line 1 is the maximum angular number.
\item line 2 is the number of radial sample that will be used
to output the eigenfunctions.
\item maximum number of overtones to be computed
\item line 4 to 6: density, vp and vs
\item line 8 and 9 frequency band in Hz
\item line 10 and 11: prefix for spheroidal and toroidal
eigenfunction files.
\end{itemize}	

%------------------------------------------------------------
\subsection{To read a single mode to ASCII format: {\tt get\_fctp}}
%------------------------------------------------------------
{\tt get\_fctp} allows you to extract a particular eigenmode form the
binary output of {\tt minosy}. The output file form {\tt get\_fctp} is
a two columns  (depth , amplitude) ASCII file.
%--------------------------------------------------------------
\subsection{Normal mode summation program: NMS}
%--------------------------------------------------------------
There are 3 input files for {\tt nms}: {\tt nms.dat  receivers.dat
sources.dat} 

{\tt nms.dat}:
\begin{alltt}
\small
 1 #time step
 2 5
 3 #number of time steps
 4 2400
 5 #source t0:
 6 1000.0
 7 #source amplitude:
 8 1.e00
 9 #source frequency band (f1,f2,f3,f4):
10 0.10E-2
11 1.00E-2
12 1.20E-2
13 1.50E-2
14 #index min, index max of overtones to be used (-1,-1 for all)
15 -1 -1
16 #geocentric correction? (T=yes, F=no)
17 T
18 #component rotation? (T=yes, F=no)
19 T
20 #S eigenfunction file prefix
21 fct_premR
22 #T eigenfunction file prefix
23 fct_premT
\end{alltt}	
with 
\begin{itemize}	
\item line 2: time step in second
\item line 4: number of time step
\item line 6: central time of the source wavelet in second. In
general, this number should be 0, otherwise your source wavelet will
be truncated. To check this number is large enough, visualize the {\tt
source.gnu} file that will be generated by {\tt nms}. If you need an
origin time at 0, you will need to shift the traces afterward.
\item line 8: amplitude of the source filter.
\item line 10 to 13: definition of the source wavelet in the 
frequency domain. The spectrum is flat between {\tt f2} and {\tt f3}
and a cosine tapper between {\tt f1} and {\tt f2} and  between {\tt f3}
and {\tt f4} is applied.
\item line 15: overtone selection. e.g. 0 0 will select only the
fundamental mode. (-1 -1 will take all the overtone)
\item line 17: do we apply geocentric correction to source and receiver
coordinate (see Dahlen and Tromp, p603)?
\item line 19:  do we apply component rotation?
\item line 21 and 23: eigenmode catalog prefix output of minosy for
spheroidal and toroidal modes respectively.
\end{itemize}	
{\tt \large sources.dat} contains all the informations about the source(s):
\begin{alltt}
\small
1 #nombre d'evenements
2 1
3 # stacking them (.true.) or not (.false.):
4 .false.
5 #n name      Mrr,     Mtt,       Mpp,     Mrt,      Mrp,       Mtp,
dep,   lat,   phi, tdelay 
6 01 M012601A  1.04e+19 -0.43e+19 -0.61e+19  2.98e+19 -2.40e+19
    0.43e+19  10.00  03.30 095.94 000.00 
\end{alltt}	
Note line 5 and 6 have been cut to fit in the page.
\begin{itemize}	
\item line 2: number of events to be computed
\item line 4: in the case of several sources, should the be stacked
in a single trace per receiver and component?
\item line 6: number, name, moment tensor, depth and location of the
event. The last number is an extra time delay that can be applied to
origin time given in {\tt nms.dat} 
\end{itemize}	
{\tt \large receivers.dat} contains all the informations about the receivers:
\begin{alltt}
\small
 1 #number of receivers
 2 11
 3 #name (4 characters), latitude, longitude in degrees
 4 109C   32.889 -117.105
 5 _AAK   42.639   74.494
 6 ACSO   40.232  -82.982
 7 AHID   42.765 -111.100
 8 _AML   42.131   73.694
 9 ANMO   34.946 -106.457
10 ANTO   39.869   32.794
11 _APE   37.069   25.531
12 _AQU   42.354   13.405
13 _ARU   56.430   58.562
14 _ARV   35.127 -118.830
\end{alltt}	
\begin{itemize}	
\item: line2 : number of receivers
\item line 4 to  number of receivers +3: name (4 characters),
latitude, longitude in degrees of each receiver.
\end{itemize}	
The outputs are 3 ASCII files for each couple source--receiver. The two
first characters of each file will be {\tt UZ}, {\tt UR} and {\tt UT}
(for vertical, radial and transverse components) if the rotation is
used and {\tt UZ}, {\tt UN} and {\tt UE} (for vertical, North and Est
components) if not. 
%=-=-=-=-=-=-=-=-=-=-=-=-=-=-=-=-=-=-=-=-=-=-=-=-=-=-=-=-=-=-=-=-=-=-=
\end{document}	
%-=-=-=-=-=-=-=-=-=-=-=-=-=-=-=-=-=-=-=-=-=-=-=-=-=-=-=-=-=-=-=-=-=-=-=

